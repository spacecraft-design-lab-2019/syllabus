\documentclass[11pt,letterpaper]{article}

\usepackage[margin=1in]{geometry}
\usepackage{termcal}
\usepackage{enumitem}
\usepackage{hyperref}
\usepackage{color}

\newcommand{\todo}[1]{\textcolor{red}{TODO: #1}}

\title{AA136/236: Spacecraft Design Capstone}
\author{Zac Manchester}
\date{Fall-Winter 2019-2020}

\begin{document}

\maketitle

\section*{Course Description}

Spacecraft design is a truly interdisciplinary subject that draws from every branch of engineering. This capstone design class brings together the material from prior classes in a way that emphasizes the interactions between disciplines and demonstrates how some of the more theoretical topics are synthesized in the practical design of a spacecraft. The class will design, build, and test a small satellite that addresses objectives and requirements posed at the beginning of the course sequence. Students will work in subsystem teams, each focusing on some aspect of the spacecraft, but exposed to many different disciplines and challenges. Practical, hands-on engineering skills will be emphasized, along with fabrication and testing of physical hardware and the creation of thorough documentation.

\section*{Instructors}

\begin{center}
\begin{tabular}{l l r}
	Prof. Zac Manchester & \textbf{Email:} \href{mailto:zacmanchester@stanford.edu}{zacmanchester@stanford.edu} & \textbf{Office:} Durand 267 \\
	CA: Max Holliday & \textbf{Email:} maholli@stanford.edu \\
	CA: Andrew Gatherer & \textbf{Email:} gatherer@stanford.edu
\end{tabular}
\end{center}

\section*{Course Objectives and Learning Outcomes}

The goal of this course is to give students hands-on experience designing and building small spacecraft subsystems and integrating them into a CubeSat. Throughout this course, students will:
\begin{enumerate}
	\item Understand how the material in many other courses needs to be integrated in the design of a system whose performance depends on the success of many interacting subsystems.
	\item Work within a small team to fabricate, and test hardware and software through rapid design iteration.
	\item Coordinate with other teams to integrate subsystems into a complete spacecraft.
	\item Gain exposure to the complete life cycle of a small satellite mission.
\end{enumerate}


\section*{Logistics}

The course will involve designing and building hardware in small teams. Class time will be used for weekly team meetings and consulting time to meet with the instructors.

\begin{itemize}
	\item All-hands meetings will be held at 4:30 on Mondays, followed by consulting hours.
	\item Lectures on selected topics will be held at 4:30 on Wednesdays, followed by consulting hours.
	\item Sub team meetings will be held once per week at times coordinated with the instructor.
	\item Attendance of weekly team meetings is mandatory.
	\item Slack will be used for coordination between teams and instructors. All students will be added to the ``SpacecraftDesignLab'' slack channel.
	\item GitHub will be used to manage project files for all teams.
\end{itemize}

\section*{Assignments and Exams}

There will be no exams in this course. Evaluation will be based on participation, contribution to design and fabrication work, and a final documentation from each team.

\section*{Fall Quarter Schedule}

\begin{enumerate}[label=\textbf{Week \arabic*:},leftmargin=3.5\parindent]
	\item Team Formation / Introduction
	\item Subsystem Requirements Definition
	\item Preliminary Design
	\item Preliminary Design
	\item Component Purchases
	\item Subsystem Fabrication
	\item Subsystem Fabrication
	\item Subsystem Testing
	\item Design Revisions
	\item Final Design
\end{enumerate}

\section*{Grading}

Grading will be based on:
\begin{itemize}
	\item 25\% Participation and attendance of team meetings
	\item 25\% Individual technical contributions quantified by git commit history and team surveys
	\item 50\% Completeness and quality of documentation
\end{itemize}
Stanford's grading system is defined by the Faculty Senate as A=Excellent, B=Good, C=Satisfactory, D=Minimal Pass, and NP=Not Passed.

\section*{References}

There is no required text for this class. However, students may wish to refer to: \textit{Spacecraft Mission Engineering: The New SMAD} by Wertz, Everett, and Pushcell.

\section*{Course Policies}

\textbf{Attendance:} This is a team-based course. In order to coordinate work among teams, participation in weekly meetings is required. If you are unable to be present at a meeting, you must notify the instructors and ensure that your teammates are prepared to present your work.

\section*{University Policies}

\textbf{The Honor Code:} It is expected that Stanford's Honor Code will be followed in all matters relating to this course. You are encouraged to meet and exchange ideas with your classmates while studying and working on homework assignments, but you are individually responsible for your own work and for understanding the material. You are not permitted to copy or otherwise reference another student's homework or computer code. If you have any questions regarding this policy, feel free to contact the professor.

Compromising your academic integrity may lead to serious consequences, including (but not limited to) one or more of the following: failure of the assignment, failure of the course, disciplinary probation, suspension from the university, or dismissal from the university.

Students are responsible for understanding the University's Honor Code policy and must make proper use of citations of sources for writing papers, creating, presenting, and performing their work, taking examinations, and doing research.

\medskip
\noindent
\textbf{Accommodations:} Students who may need an academic accommodation based on the impact of a disability must initiate the request with the Office of Accessible Education (OAE). Professional staff will evaluate the request with required documentation, recommend reasonable accommodations, and prepare an Accommodation Letter for faculty dated in the current quarter in which the request is being made. Students should contact the OAE as soon as possible since timely notice is needed to coordinate accommodations.




\end{document}
